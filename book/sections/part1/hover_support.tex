The time is now! We're finally adding a real feature to our LSP! We'll start by
adding hover support for instructions. A few things need to happen first though,
before we see that sweet sweet hover documentation window. First we need some 
source of information for our instructions. This source needs to be in a fairly
organized, regular format fit for parsing. After finding such a source, we then
need to write some code to parse and load it into our program. This will require
some parsing code (duh), as well as some data structures to represent our information
once it's loaded from disk. Finally, we need to trigger every time a hover request
comes in from the client, check our relevant data structures, and return any results
we find in the right format. Without further ado, let's get started.

\subsection{Finding an Information Source}


\subsection{Data Structures and Parsing}


\subsection{Hover Requests}



\subsection{Cleanup}

% Motivate having info

% Introduce the instruction info repo

% Introduce the register xml files from asm-lsp

% Sketch out structs for instructions and registers

% Quickly go over xml parsing functions

% Put in start up code

% Walk through the hover request/ response interaction in detail
    % go to spec
    % sketch out according code
    % Make improvements
